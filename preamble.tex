% !TEX root = main.tex
% \documentclass[logo, 12pt, a4paper, usenames, dvipsnames, twopage, oneside]{book}
\documentclass[logo, 12pt, a4paper, usenames, dvipsnames, twopage, twoside]{book}

\usepackage{silence}
\WarningsOff*
\ErrorsOff*


\usepackage[utf8]{inputenc}
\usepackage[brazil]{babel}
\usepackage{soulutf8}


\usepackage[T1]{fontenc}

\usepackage{mathptmx}

\usepackage{bm}


\newcommand{\bcal}[1]{\bm{\mathcal{#1}}}

% \usepackage{commath}



%   --------------------
%   ----- Chapters -----
%   --------------------
% \usepackage{lmodern}
% \usepackage{graphicx}
% \usepackage{xcolor}
% \usepackage{titlesec}
% \usepackage{microtype}

% \definecolor{myblue}{RGB}{0,82,155}

% \titleformat{\chapter}[display]
%   {\normalfont\bfseries\color{myblue}}
%   {\filleft\hspace*{-60pt}%
%     \rotatebox[origin=c]{90}{%
%       \normalfont\color{black}\Large%
%         \textls[180]{\textsc{\chaptertitlename}}%
%     }\hspace{10pt}%
%     {\setlength\fboxsep{0pt}%
%     \colorbox{myblue}{\parbox[c][3cm][c]{2.5cm}{%
%       \centering\color{white}\fontsize{80}{90}\selectfont\thechapter}%
%     }}%
%   }
%   {10pt}
%   {\titlerule[2.5pt]\vskip3pt\titlerule\vskip4pt\LARGE\sffamily}




\newcommand\footnoteref[1]{\textsuperscript{\ref{#1}}}




\usepackage[explicit]{titlesec}
\usepackage[many]{tcolorbox}
\usepackage{lmodern}
\usepackage{lipsum}

\definecolor{titlebgdark}{RGB}{80,80,80}
% \definecolor{titlebgdark}{RGB}{0, 68, 102}
% \definecolor{titlebgdark}{RGB}{0,163,243}

\definecolor{titlebglight}{RGB}{200,200,200}
% \definecolor{titlebglight}{RGB}{0, 102, 153}
% \definecolor{titlebglight}{RGB}{191,233,251}

% \titleformat{\chapter}[display]
%   {\normalfont\huge\bfseries}
%   {}
%   {20pt}
%   {%
%     \begin{tcolorbox}[
%       enhanced,
%       colback=titlebgdark,
%       boxrule=0.25cm,
%       colframe=titlebglight,
%       arc=0pt,
%       outer arc=0pt,
%       leftrule=0pt,
%       rightrule=0pt,
%       fontupper=\color{white}\sffamily\bfseries\huge,
%       enlarge left by=-1in-\hoffset-\oddsidemargin,
%       enlarge right by=-\paperwidth+1in+\hoffset+\oddsidemargin+\textwidth,
%       width=\paperwidth,
%       left=1in+\hoffset+\oddsidemargin,
%       right=\paperwidth-1in-\hoffset-\oddsidemargin-\textwidth,
%       top=0.6cm,
%       bottom=0.6cm,
%       overlay={
%         \node[
%           fill=titlebgdark,
%           draw=titlebglight,
%           line width=0.15cm,
%           inner sep=0pt,
%           text width=1.7cm,
%           minimum height=1.7cm,
%           align=center,
%           font=\color{white}\sffamily\bfseries\fontsize{30}{36}\selectfont
%         ]
%         (chapname)
%         at ([xshift=-4cm]frame.north east)
%         {\thechapter};
%         \node[font=\small,anchor=south,inner sep=2pt] at (chapname.north)
%         {\MakeUppercase\chaptertitlename};
%       }
%     ]
%     #1
%     \end{tcolorbox}%
%   }
% \titleformat{name=\chapter,numberless}[display]
%   {\normalfont\huge\bfseries}
%   {}
%   {20pt}
%   {%
%     \begin{tcolorbox}[
%       enhanced,
%       colback=titlebgdark,
%       boxrule=0.25cm,
%       colframe=titlebglight,
%       arc=0pt,
%       outer arc=0pt,
%       remember as=title,
%       leftrule=0pt,
%       rightrule=0pt,
%       fontupper=\color{white}\sffamily\bfseries\huge,
%       enlarge left by=-1in-\hoffset-\oddsidemargin,
%       enlarge right by=-\paperwidth+1in+\hoffset+\oddsidemargin+\textwidth,
%       width=\paperwidth,
%       left=1in+\hoffset+\oddsidemargin,
%       right=\paperwidth-1in-\hoffset-\oddsidemargin-\textwidth,
%       top=0.6cm,
%       bottom=0.6cm,
%     ]
%     #1
%     \end{tcolorbox}%
%   }
% \titlespacing*{\chapter}
%   {0pt}{0pt}{40pt}
% \makeatother







%   --------------------
%   ----- Sections -----
%   --------------------
% \usepackage[explicit]{titlesec}
\usepackage{soul}

\definecolor{titleblue}{RGB}{0, 68, 102}
\definecolor{titlegray}{RGB}{80,80,80}
% \definecolor{titleblue}{RGB}{0,163,243}
% \definecolor{titleblue}{RGB}{54, 90, 124}

% \newbox\TitleUnderlineTestBox
% \newcommand*\TitleUnderline[1]
%   {%
%     \bgroup
%     \setbox\TitleUnderlineTestBox\hbox{\colorbox{titleblue}\strut}%
%     \setul{\dimexpr\dp\TitleUnderlineTestBox-.3ex\relax}{.3ex}%
%     \ul{#1}%
%     \egroup
%   }
% \newcommand*\SectionNumberBox[1]
%   {%
%     \colorbox{titlegray}
%       {%
%         \makebox[2.5em][c]
%           {%
%             \color{white}%
%             \strut
%             \csname the#1\endcsname
%           }%
%       }%
%     \TitleUnderline{\ \ \ }%
%   }
% \titleformat{\section}
%   {\Large\bfseries\sffamily\color{titlegray}}
%   {\SectionNumberBox{section}}
%   {0pt}
%   {\TitleUnderline{#1}}
% \titleformat{\subsection}
%   {\large\bfseries\sffamily\color{titlegray}}
%   {\SectionNumberBox{subsection}}
%   {0pt}
%   {\TitleUnderline{#1}}















%   -------------------------------------------------
%   ----- Definitions, Assumptions and Examples -----
%   -------------------------------------------------
\usepackage[many]{tcolorbox}
\usetikzlibrary{calc}

\definecolor{myblue}{RGB}{50, 130, 180}
\definecolor{mygreen}{RGB}{50, 180, 130}
\definecolor{mygray}{RGB}{80,80,80}
\definecolor{mygray2}{RGB}{40,40,40}

\tcbset{mystyle/.style={
  breakable,
  enhanced,
  outer arc=0pt,
  arc=0pt,
  colframe=mygreen!80,
  colback=mygreen!20,
  valign=center,
  every float=\centering,
  add to natural height=1cm,
  width=\linewidth-2mm,
  before=,
  after=\hfill,
  before skip=1.0cm,
  after skip=1.0cm,
  attach boxed title to top left,
  boxed title style={
    colback=mygreen!80,
    outer arc=0pt,
    arc=0pt,
    top=3pt,
    bottom=3pt,
    },
  fonttitle=\sffamily
  }
}

\newtcolorbox[auto counter,number within=section]{definition}[1][]{
  mystyle,
  title=Definição~\thetcbcounter,
  overlay unbroken and first={
      \path
        let
        \p1=(title.north east),
        \p2=(frame.north east)
        in
        node[anchor=west,font=\sffamily,color=mygreen,text width=\x2-\x1]
        at (title.east) {#1};
  }
}
\newtcolorbox[auto counter]{assumption}[1][]{
  mystyle,
  colback=white,
  rightrule=0pt,
  toprule=0pt,
  title=Suposição SLR.\thetcbcounter,
  overlay unbroken and first={
      \path
        let
        \p1=(title.north east),
        \p2=(frame.north east)
        in
        node[anchor=west,font=\sffamily,color=mygray2,text width=\x2-\x1]
        at (title.east) {#1};
  }
}


\tcbuselibrary{skins,breakable}
\newcounter{example}


\colorlet{colexam}{black!20!RoyalBlue}
\newtcolorbox[use counter=example]{example}{%
empty,title={Exemplo \thetcbcounter},attach boxed title to top left,
boxed title style={empty,size=minimal,toprule=2pt,top=4pt,
overlay={\draw[colexam,line width=2pt]
([yshift=-1pt]frame.north west)--([yshift=-1pt]frame.north east);}},
coltitle=colexam,fonttitle=\Large\bfseries,
before=\par\medskip\noindent,parbox=false,boxsep=0pt,left=0pt,right=3mm,top=4pt,
breakable,pad at break*=0mm,vfill before first,
overlay unbroken={\draw[colexam,line width=1pt]
([yshift=-1pt]title.north east)--([xshift=-0.5pt,yshift=-1pt]title.north-|frame.east)
--([xshift=-0.5pt]frame.south east)--(frame.south west); },
overlay first={\draw[colexam,line width=1pt]
([yshift=-1pt]title.north east)--([xshift=-0.5pt,yshift=-1pt]title.north-|frame.east)
--([xshift=-0.5pt]frame.south east); },
overlay middle={\draw[colexam,line width=1pt] ([xshift=-0.5pt]frame.north east)
--([xshift=-0.5pt]frame.south east); },
overlay last={\draw[colexam,line width=1pt] ([xshift=-0.5pt]frame.north east)
--([xshift=-0.5pt]frame.south east)--(frame.south west);},%
}




\usepackage{mathtools}
\usepackage{empheq}
\usepackage{eucal}
\usepackage{amsfonts}
\usepackage{mathrsfs}
\usepackage{newtxmath}



% ----- Font Effect of MathAlfa with Boondox to Cal -----
\DeclareFontFamily{U}{BOONDOX-calo}{\skewchar\font=45 }
\DeclareFontShape{U}{BOONDOX-calo}{m}{n}{
  <-> s*[1.05] BOONDOX-r-calo}{}
\DeclareFontShape{U}{BOONDOX-calo}{b}{n}{
  <-> s*[1.05] BOONDOX-b-calo}{}
\DeclareMathAlphabet{\mathcalboondox}{U}{BOONDOX-calo}{m}{n}
\SetMathAlphabet{\mathcalboondox}{bold}{U}{BOONDOX-calo}{b}{n}
\DeclareMathAlphabet{\mathbcalboondox}{U}{BOONDOX-calo}{b}{n}
% -------------------------------------------------------



% \usepackage[top=3cm, left=3cm, bottom=2cm, right=2cm]{geometry}
\usepackage[top=2.5cm, left=2.5cm, bottom=2.5cm, right=2.5cm]{geometry}
\usepackage{fancyhdr}
\usepackage{enumitem}

\usepackage{booktabs}
\setlength\heavyrulewidth{0.35ex}
\setlength\lightrulewidth{0.25ex}

% \usepackage[nottoc,numbib]{tocbibind}
\usepackage[nottoc]{tocbibind}

\usepackage{graphicx}
\usepackage{xcolor}
\usepackage{subfig}
\usepackage{subcaption}
%\usepackage{varioref}
\usepackage[colorlinks]{hyperref}
\usepackage{cleveref}
% \usepackage{indentfirst}


\usepackage{mwe}
\usepackage{subcaption}


\usepackage{caption}
\captionsetup[table]{skip=10pt}

\usepackage[bottom]{footmisc}

\usepackage{dcolumn}
\usepackage{tabu}

\usepackage{float}

\usepackage{algorithm}
% \SetKwFor{Para}{para}{fa\c{c}a}{fim para}
\usepackage{algcompatible}
\usepackage[noend]{algpseudocode}
\usepackage{framed}
\floatname{algorithm}{Algoritmo}

\newcounter{algsubstate}
\renewcommand{\thealgsubstate}{\alph{algsubstate}}
\newenvironment{algsubstates}
  {\setcounter{algsubstate}{0}%
   \renewcommand{\State}{%
     \stepcounter{algsubstate}%
     \Statex {\footnotesize\thealgsubstate:}\space}}
  {}


\usepackage{upquote}

\DeclarePairedDelimiter\abs{\lvert}{\rvert}%
\DeclarePairedDelimiter\norm{\lVert}{\rVert}%

\makeatletter
\let\oldabs\abs
\def\abs{\@ifstar{\oldabs}{\oldabs*}}
%
\let\oldnorm\norm
\def\norm{\@ifstar{\oldnorm}{\oldnorm*}}
\makeatother



\DeclarePairedDelimiter\set\{\}

\makeatletter
\newcommand{\algmargin}{\the\ALG@thistlm}
\makeatother
\newlength{\whilewidth}
\settowidth{\whilewidth}{\algorithmicwhile\ }
\algdef{SE}[parWHILE]{parWhile}{EndparWhile}[1]
  {\parbox[t]{\dimexpr\linewidth-\algmargin}{%
     \hangindent\whilewidth\strut\algorithmicwhile\ #1\ \algorithmicdo\strut}}{\algorithmicend\ \algorithmicwhile}%
\algnewcommand{\parState}[1]{\State%
  \parbox[t]{\dimexpr\linewidth-\algmargin}{\strut #1\strut}}

\newcommand\sbullet[1][.5]{\mathbin{\vcenter{\hbox{\scalebox{#1}{$\bullet$}}}}}

\usepackage{libertine}
\usepackage{libertinust1math}
\usepackage[T1]{fontenc}

\usepackage{array}
\usepackage{makecell}

\renewcommand\theadalign{bc}
\renewcommand\theadfont{\bfseries}
\renewcommand\theadgape{\Gape[4pt]}
\renewcommand\cellgape{\Gape[4pt]}

%\usepackage{cite}

%\setlogooptions{width=2cm}
\usepackage{graphicx}
\usepackage{mwe}       % for the example logo
\usepackage{blindtext} % tor the example text

\hypersetup{
        hidelinks,
        %backref=true,
        %pagebackref=true,
        %hyperindex=true,
        breaklinks=true,
        colorlinks=true,
        %linkcolor=OliveGreen,
        linkcolor=Blue,
        citecolor=Blue,
        urlcolor=blue,
        %bookmarks=true,
        bookmarksopen=false,
        pdftitle={Title},
        pdfauthor={Author}
        }


% for adjustwidth environment
\usepackage[strict]{changepage}

% for formal definitions
\usepackage{framed}

% environment derived from framed.sty: see leftbar environment definition
\definecolor{formalshade}{rgb}{0.90,0.98,1}
\definecolor{formalshadei}{rgb}{0.88,1,0.90}
% \definecolor{formalshade}{rgb}{0.95, 0.95, 0.95}
\definecolor{citeblue}{rgb}{0.2,0.2,0.7}
\definecolor{citegreen}{rgb}{0.2,0.7,0.2}
% \definecolor{midnightgray}{rgb}{0.1, 0.1, 0.1}

\newenvironment{formal}{%
  \def\FrameCommand{%
    \hspace{1pt}%
    {\color{citegreen}\vrule width 2pt}%
    {\color{formalshadei}\vrule width 4pt}%
    \colorbox{formalshadei}%
  }%
  \MakeFramed{\advance\hsize-\width\FrameRestore}%
  \noindent\hspace{-4.55pt}% disable indenting first paragraph
  \begin{adjustwidth}{}{7pt}%
  \vspace{2pt}\vspace{2pt}%
}
{%
  \vspace{2pt}\end{adjustwidth}\endMakeFramed%
}


%   --- BILIOGRAPGY ---

\usepackage[authoryear,square]{natbib}

\AtBeginDocument{
  \let\natbibbibitem\bibitem
  \let\bibitem\valientbibitem
}
\makeatletter
\def\valientbibitem[#1]#2{%
  \natbibbibitem[#1]{#2}%
  \begingroup\let\citename\@firstofone
  [#1]
  \endgroup
}
\makeatother


% \usepackage[alf,abnt-emphasize=bf]{abntex2cite}




\newcommand*{\StateSpace}[1]{\mathcal{#1}}

\newcommand*{\myvar}[1]{%
   \mathord{\mbox{%
      \sffamily\bfseries\itshape
      #1%
   }}%
}





\makeatletter
\renewcommand{\ALG@name}{Algoritmo}
\renewcommand{\listalgorithmname}{Lista de \ALG@name s}
\makeatother




% % Declaracoes em Português
% \algrenewcommand\algorithmicend{\textbf{fim}}
% \algrenewcommand\algorithmicdo{\textbf{faça}}
% \algrenewcommand\algorithmicwhile{\textbf{enquanto}}
% \algrenewcommand\algorithmicfor{\textbf{para}}
% \algrenewcommand\algorithmicif{\textbf{se}}
% \algrenewcommand\algorithmicthen{\textbf{então}}
% \algrenewcommand\algorithmicelse{\textbf{senão}}
% \algrenewcommand\algorithmicreturn{\textbf{devolve}}
% \algrenewcommand\algorithmicfunction{\textbf{função}}

% % Rearranja os finais de cada estrutura
% \algrenewtext{EndWhile}{\algorithmicend\ \algorithmicwhile}
% \algrenewtext{EndFor}{\algorithmicend\ \algorithmicfor}
% \algrenewtext{EndIf}{\algorithmicend\ \algorithmicif}
% \algrenewtext{EndFunction}{\algorithmicend\ \algorithmicfunction}

% % O comando For, a seguir, retorna 'para #1 -- #2 até #3 faça'
% \algnewcommand\algorithmicto{\textbf{até}}
% \algrenewtext{For}[3]%
% {\algorithmicfor\ #1 $\gets$ #2 \algorithmicto\ #3 \algorithmicdo}




\renewcommand{\baselinestretch}{1.075}
% \renewcommand{\baselinestretch}{1.0}
\setlength{\parindent}{2em}
% \setlength{\parskip}{0.5em}




\usepackage{nomencl}
\makenomenclature
\setlength{\nomlabelwidth}{2.5cm}
\def\pagedeclaration#1{\dotfill\nobreakspace#1}
% \def\nomentryend{.}
\def\nomlabel#1{\textbf{#1}\hfil}

\renewcommand{\nomname}{Lista de Abreviações}




% Fancy epigraphs
\usepackage{times}
\usepackage{xcolor}
\usepackage{colortbl}

\definecolor{mColor1}{rgb}{0.9,0.9,0.9}
\newcolumntype{O}{>{\columncolor{mColor1}}}
\newcolumntype{D}[1]{>{\raggedright\arraybackslash}m{#1}}
\newcolumntype{E}[1]{>{\centering\arraybackslash}m{#1}}
\newcolumntype{N}{@{}>{\columncolor{white}[0pt][0pt]}m{0pt}@{}}


% \usepackage{microtype}

% \newcommand*{\belowrulesepcolor}[1]{%
%   \noalign{%
%     \kern-\belowrulesep
%     \begingroup
%       \color{#1}%
%       \hrule height\belowrulesep
%     \endgroup
%   }%
% }
% \newcommand*{\aboverulesepcolor}[1]{%
%   \noalign{%
%     \begingroup
%       \color{#1}%
%       \hrule height\aboverulesep
%     \endgroup
%     \kern-\aboverulesep
%   }%
% }




\usepackage{epigraph}
\usepackage{tabto}
\usepackage{ifthen}
\setlength{\epigraphwidth}{0.75\textwidth}
\setlength{\afterepigraphskip}{\baselineskip}
\setlength{\beforeepigraphskip}{0.4\baselineskip}
\definecolor{faded}{RGB}{230,230,230}
\makeatletter
\newenvironment{@leftepigraph}{
   \setlength\topsep{0pt}
   \setlength\parskip{0pt}
   \renewcommand{\epigraphflush}{flushleft}
}{}
\newenvironment{@rightepigraph}{
   \setlength\topsep{0pt}
   \setlength\parskip{0pt}
   \renewcommand{\epigraphflush}{flushright}
   \renewcommand{\sourceflush}{flushright}
   \renewcommand{\textflush}{flushright}
}{}
\newcommand{\@bgquote}[2]{\tabto*{#1}{\fontsize{100pt}{0pt}\selectfont{}\color{faded}\smash{\raisebox{-60pt}{#2}}}\tabto*{0pt}}
\newcommand{\@generalquote}[6]{
\begin{#6}
\epigraph{\@bgquote{#5}{#4}#3}{\textsc{#2}\ifthenelse{\equal{#1}{}}{}{\\\textit{#1}}}
\end{#6}}
\newcommand{\lquote}[3][]{\@generalquote{#1}{#2}{#3}{''}{\dimexpr(\epigraphwidth-50pt)}{@leftepigraph}}
\newcommand{\rquote}[3][]{\@generalquote{#1}{#2}{#3}{``}{0pt}{@rightepigraph}}
\makeatother














\usepackage{tikz}
\usepackage[most]{tcolorbox}
\definecolor{ocre}{RGB}{52,177,201}
\definecolor{ultramarine}{RGB}{0,45,97}
\definecolor{mybluei}{RGB}{0,173,239}
\definecolor{myblueii}{RGB}{63,200,244}
\definecolor{myblueiii}{RGB}{199,234,253}

\definecolor{mygreeni}{RGB}{40, 115, 20}
\definecolor{mygreenii}{RGB}{50, 150, 50}
\definecolor{mygreeniii}{RGB}{55, 175, 55}


% \usepackage{fourier}
\usepackage[explicit,calcwidth]{titlesec}

\renewcommand\thechapter{\arabic{chapter}}

\newcommand\chapnumfont{%
  \fontsize{120}{130}\color{mygreeniii}\selectfont%
}

\newcommand\chapnamefont{%
  \normalfont\color{white}\scshape\small\bfseries
}

\titleformat{\chapter}
  {\normalfont\huge\filleft}
  {}
  {0pt}
  {\stepcounter{chapshift}%
  \begin{tikzpicture}[remember picture,overlay]
  \fill[mygreenii]
    (current page.north west) rectangle ([yshift=-6.5cm]current page.north east);
  \node[
      fill=mygreeni,
      text width=2\paperwidth,
      rounded corners=5cm,
      text depth=8cm,
      anchor=center,
      inner sep=0pt] at (current page.north east) (chaptop)
    {\thechapter};%
  \node[
      anchor=south east,
      inner sep=20pt,
      outer sep=0pt] (chapnum) at ([xshift=-20pt]chaptop.south)
    {\chapnumfont\thechapter};
  \node[
      anchor=south,
      inner sep=0pt] (chapname) at ([yshift=2pt]chapnum.south)
  {};
  \node[
      anchor=north east,
      align=right,
      font=\sffamily\color{white}\fontsize{25}{50}\selectfont,
      inner xsep=0pt] at ([yshift=-0.5cm]chapname.east|-chapnum.south)
  {\parbox{1.3\textwidth}{\raggedleft#1}};
  \end{tikzpicture}%
  }

\titleformat{name=\chapter,numberless}
  {\normalfont\huge\filleft}
  {}
  {0pt}
  {\begin{tikzpicture}[remember picture,overlay]
  \fill[mygreenii]
    (current page.north west) rectangle ([yshift=-5.0cm]current page.north east);
  \node[
      fill=mygreeni,
      text width=2\paperwidth,
      rounded corners=3cm,
      text depth=5cm,
      anchor=center,
      inner sep=0pt] at (current page.north east) (chaptop)
    {};%
  \node[
      anchor=south east,minimum width=2in,
      inner sep=10pt,
      outer sep=0pt] (chapnum) at ([xshift=-20pt]chaptop.south)
    {};
  \node[
      anchor=south,
      inner sep=0pt] (chapname) at ([yshift=2pt]chapnum.south)
  {};
  \node[
      anchor=north east,
      align=right,
      font=\sffamily\color{white}\fontsize{25}{50}\selectfont,
      inner xsep=0pt] at ([yshift=-0.5cm]chapname.east|-chapnum.south)
  {\parbox{1.0\textwidth}{\raggedleft#1}};
  \end{tikzpicture}%
  }

\titlespacing*{\chapter}{0pt}{0pt}{4.0cm}
\titlespacing*{name=\chapter,numberless}{0pt}{0pt}{4.0cm}

\titleformat{\section}
  {\addtolength{\titlewidth}{4pc}\normalfont\Large\sffamily\bfseries}
  {\llap{\colorbox{mygreenii}{\parbox{2cm}{\strut\color{white}\hfill\thesection}}\hspace{20pt}}}
  {0em}{#1}
  [{\titleline*[l]{\hspace*{\dimexpr-2cm-20pt-2\fboxsep\relax}\color{mygreenii}\titlerule[1.5pt]}}]
\titleformat{\subsection}
  {\addtolength{\titlewidth}{2pc}\normalfont\large\sffamily}
  {\llap{\colorbox{mygreenii}{\parbox{2cm}{\strut\color{white}\hfill\thesubsection}}\hspace{20pt}}}
  {0em}{#1}
  [{\titleline*[l]{\hspace*{\dimexpr-2cm-20pt-2\fboxsep\relax}\color{mygreenii}\titlerule[1.5pt]}}]

\usetikzlibrary{calc}
\pagestyle{plain}

\newcounter{chapshift}
\addtocounter{chapshift}{-1}

\newcommand\BoxColor{mygreenii}

\def\subsectiontitle{}
% \renewcommand{\sectionmark}[1]{\markright{\sffamily\normalsize#1}{}}
% \renewcommand{\subsectionmark}[1]{\def\subsectiontitle{#1}}





% \usepackage{etoolbox,fancyhdr}

% \pagestyle{fancy}

% \renewcommand{\headrule}{{\color{myblueii}%
% \hrule width\headwidth height\headrulewidth depth\headrulewidth}}

% \renewcommand{\chaptermark}[1]{\markboth{\sffamily\normalsize\bfseries \ #1}{}}
% \renewcommand{\sectionmark}[1]{\markright{\sffamily\normalsize#1}{}}
% \fancyhf{} \fancyhead[LE,RO]{\sffamily\normalsize\colorbox{myblueii}{\color{white}\sffamily\bfseries\strut\quad\thepage\quad}}
% \fancyhead[LO]{\textcolor{mybluei} \rightmark%
% \begin{tikzpicture}[overlay,remember picture]
%   \node[fill=\BoxColor,inner sep=-1pt,rectangle,text width=1cm,
%     text height=28cm,align=center,anchor=north east]
%   at ($ (current page.north east) $)
%   {\rotatebox{90}{\parbox{18cm}{%
%    \centering\textcolor{white}{\bfseries\scshape\thechapter.\leftmark \hspace{2cm}\rightmark\hspace{2cm}\thesubsection~\subsectiontitle}}}};
%   \end{tikzpicture}}
% \fancyhead[RE]{\textcolor{mybluei}\leftmark%
%   \begin{tikzpicture}[overlay,remember picture]
%   \node[fill=\BoxColor,inner sep=0pt,rectangle,text width=1cm,
%     text height=28cm,align=center,anchor=north west]
%   at ($ (current page.north west) $)
%   {\rotatebox{90}{\parbox{18cm}{%
%     \centering\textcolor{white}{\bfseries\scshape\thechapter.\leftmark \hspace{2cm}\rightmark\hspace{2cm}\thesubsection~\subsectiontitle}}}};
%   \end{tikzpicture}}
% \renewcommand{\headrulewidth}{.5pt}
% \addtolength{\headheight}{2.5pt}
% \newcommand{\footrulecolor}[1]{\patchcmd{\footrule}{\hrule}{\color{#1}\hrule}{}{}}
% \renewcommand{\headrulewidth}{.5pt}
% \addtolength{\headheight}{2.5pt}
% \renewcommand{\footrulewidth}{.5pt}
% \fancyfoot[LE,RO]{\footnotesize\bfseries\itshape Foot 1}
% \fancyfoot[C]{\footnotesize\bfseries Foot 2}
% \fancyfoot[RE,LO]{\footnotesize\bfseries\itshape Foot 3}

% \fancypagestyle{plain}{%
%   \fancyhf{}%
%   \renewcommand{\headrulewidth}{0pt}
%   \renewcommand{\footrulewidth}{0pt}
% }





\makeatletter
\renewcommand{\cleardoublepage}{
\clearpage\ifodd\c@page\else
\hbox{}
\vspace*{\fill}
\thispagestyle{empty}
\newpage
\fi}
\makeatother












% \title{\textbf{Generative Adversarial Networks for\\\vspace{0.5cm} Semi-Blind Audio Source Separation}}
\title{Generative Adversarial Networks for Semi-Blind Audio Source Separation}
\author{Tito Caco Curimbaba Spadini}
\date{2019}
